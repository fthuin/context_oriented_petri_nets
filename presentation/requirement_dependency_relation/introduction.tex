\begin{frame}[noframenumbering]
	\frametitle{Requirement dependency relation}

	\textbf{Example} A system with two contexts:

	\begin{itemize}
		\item NLBS context (N)
		\item Connectivity context (C)
	\end{itemize}

	The position calculation of the NLBS service is based on the location inferred
	from a local network connection. The source context NLBS \emph{requires} the
	target context Connectivity.

	\begin{exampleblock}{Relation type}
		This relation between an activation of a context and an already active context
		is called a \emph{requirement dependency relation},
		denoted \tikz{\scriptsize
		\node (A) {N}; \node[right of = A] (B) {Pos}; \path[-triangle 60 reversed] (A) edge node {} (B);
		}
	\end{exampleblock}
\end{frame}
